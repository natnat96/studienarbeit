\documentclass[a4paper,12pt, german]{report}
\usepackage[german]{babel}
\usepackage[utf8]{inputenc}
\setcounter{tocdepth}{2}

\usepackage{acro}

\DeclareAcronym{ki}{
  short = KI ,
  long  = künstliche Intelligenz
}



\usepackage{graphicx}
\usepackage{subcaption}
\usepackage{pdflscape}
\usepackage{array}
\usepackage{hhline,float}


\begin{document}
\title{Potentiale beim Einsatz künstlicher Intelligenz in der Produktion}
\author{Nathalie Becker}


\begin{titlepage}
\maketitle
\end{titlepage}


\begin{center}
\textbf{Abstract}
\end{center}
This is the Abstract


\tableofcontents
\printacronyms

\chapter{Einleitung}

Schon seit Jahren sind informatische Errungenschaften und Wirtschaftsbereiche wie die Produktion eng miteinander verknüpft. Längst haben die Unternehmen begriffen, dass der Einsatz von Softwarelösungen einen postiven Effekt auf die Produtivität und die Wirschaftlichkeit haben.
Das Thema der künstlichen Intelligenz ist momentan/schon seit Jahren eines der größten Forschungsfelder in der Informatik. 


\chapter{Stand der Technik}

\section{Künstliche Intelligenz}

Der Forschungsbereich der künstlichen Intelligenz ist keineswegs neu... . Eine erste Anwendung von KI ist der Schachcomputer, der (dtaum) einen Schachweltmeister besiegte. Das war durch ... möglich. Die heutigen Anwendungen und Forschungsbereiche gehen weit über die Berechnung von Spielzügen hinaus. Im folgenden Abschnitt wird der Begriff der künstlichen Intelligenz näher erläutert. \cite{01}


\subsection{Grundlagen}

% Zwiebeldarstellung von KI
% Imitations der Menschlichen Intelligenz...

Eine tatsächliche Definition von künstlicher Intelligenz ist nicht einfach. Das Problem liegt hierbei bei der Definiton der Wortes "Intelligenz"...
Die Mehrheit der Literatur geht von der menschlichen Intelligenz aus, die also künstlich nachgebaut werden soll. 
%Was sind die Bestandteile von Intelligenz
Einer der wichtigsten Punkte, die menschliche Intelligenz ausgmacht ist das Treffen von Entscheidungen. \cite{01} %gerne nach andere Literatur hier

---Sieben Segmente künstlicher Intelligenz (Maschineller lernen, Robotik, Kybernetik,Expertensysteme, Verarbeitung natürlicher Sprachen, industrielle Bildverarbeitung, Spracherkennung)??( https://www.arnold-it.com/digitalisierung/kuenstliche-intelligenz-ki/)---

---Teilgebiete aufgeteilt in Handeln, Wahrnhemen und lernen, je 4 untergebiete (https://www.marketinginstitut.biz/blog/kuenstliche-intelligenz/) ---



\subsubsection{Starke und Schwache KI}




\subsubsection{Maschinelles Lernen}

Ein Segment von künstlicher Intelligenz ist das maschinelle Lernen. 

% Zwiebeldarstellung mit maschinellem lernen, depp learning und neuronalen netzen
% Was ist maschinelles Lernen
% 3 Arten von Maschinellem Lernen (überwachtes lernen, unüberwachtes lernen, teilüberwachtes lernen und verstärkendens lernen (Abbildung 01: https://www.arnold-it.com/digitalisierung/machine-learning-maschinelles-lernen/) 


\subsection{Künstliche Intelligenz in der Wirtschaft}

\subsubsection{zeitliche Entwicklung}

??? 
Trendsonar der OEFIT

\subsubsection{Zugänglichkeit der KI für Unternehmen}

\subsubsection{Anwendungsbereiche in der Wirtschaft}

\paragraph{Mensch-Maschine}

% Erweiterung der Prodktionsfunktion
% Zugänglichkeit von KI ( 

\paragraph{Anwendungen}

% Wo und in welchen Unternhemen wird KI für was verwendet

\paragraph{Wachstum}

%Was nimmt Ki für eine Rolle ein: in den letzten Jahren, aktuell, in Zukunft

%Gefahr für Arbeitnehmer



\section{Produktion}

- Prodktion eine der Grundfunktionen eines Industrieunternhemens, nebenBaschaffung und Absatz; abr auch Funanzierung, Forschung unsd Entwcklung, Personal 
- Unternehmensfuntionen je nach Unternehmen unterschiedlich gegliedert; verschiedene Schnittstellen
- Abgrenzung zu Logistik: keine Prod-Logistik, Koordination des Materialflusses und des Informationsflusses zum Wertschöpfungsprozess vom Zulierferer zum Kunden
\cite{07}




\subsection{Grundlagen Teilbereiche der Produktion}

Gelenkter Einsatz von Proktionsfaktoren zur Herstellung von Gütern oder Erzeugung von Dienstleistungen. Hauptaspekte für erfolgreiche Prosuktion, dh Erreichung der Produktionsziele: Planung, Organisation, Steuerung und Überwachung 
Abbildung: Produktion als Kombinationsprozess: Input -° Throughput -° Output (Abb 1.2)
Abbildung: Produktion im betrieblichen Umfeld (abb1.3)
Produktionsprogramm: bestimmt in welcher Menge und art Produktionsfaktoren eingesetzt/beschafft werden müssen.
Produktion muss finanziert/Investiert werden: Einzahlung erst nach Auszahlung; Anschaffung von Betriebsmitteln (Grundsctück,Gebäude,...)
\cite{07}

HäuftigZiel der Prudktion: monetäre (Gewinn, Umsatz, Maximierung EK,... Kostenminimierung)und nicht-monetäre Ziele (weitere Unterteilung quantitative: Maximierung Marktanteil, Wachstum, Prouktivität / nicht - quanititative Ziele: Sicherung Arbeitsplätze, Vermeidung von Umweltlasten) 
Ökonomische (Gweinn, Kosten, Wirtschaftlichkeit) und nicht ökonomische Ziele (technisch orientierte Ziele: Menge, Qualität, Termine; human- und soziealorientierte Ziele)
Wirtschaftlichkeit ist eine wichtige ökonomische Zielgröße: Verhältnis Output zu Input
Produktivität Verhältnis Ausbringungsmeneg nud Faktoreinsatz (nur eine Produk- und Faktorart) (tech.Ziel)

"Zusammenfassend können als typische Ziele der Produktionswirtschaft geringe Kosten,
ein hoher Output, eine hohe Produktqualität, eine hohe Termineinhaltung und Auslastung
der Fertigungsbereiche (oder einzelner Maschinen) sowie geringe Durchlaufzeiten
genannt werden." Die Ziele fallen in unterschiedliche Planungsbereiche

\subsection{Produktionsfaktoren}

Alles, was zur Erstellung der Sach- und dienstleistungen dient, aber auch zur Aufrechterhaltung und Ausbau der Leistungsbereitschaft
In VWL: strak aggregiert-°Arbeit, Boden und Kapital (oder nur Arbeit und Kapital)
In BWL; Abb 1.4 Produktionsfaktorsystem nach Gutenberg; menschl.Arbeit, Betriebsmittel und Werkstoffe (Seite 6 ist alles genau erklärt)
Tab 2.1 Produktionstheorie

\subsection{Produktionsplanung und -steuerung}



\chapter{Aktuelle Anwendungen von künstlicher Intelligenz in der Produktion}

% Hier Beispiele über zb Qualitätssicherung, Maintenace,..

% Menschintegrierend, nutzung oder alleine KI/Roboter

% Skalierbarkeit

% Analystische Anwendung, agierende Anwedungen mit Robotern

%Vorgehen zur Entwicklung von KI

\section{Qualitätssicherung}
% Platinenprüfung

\section{Predictive Maintantance}


\section{Produktionsroboter}
% Erkennt, wie Gegestand liegt und kann es aufnhmen und zum nächsten Punkt richtig einstzen



\chapter{Potentiale für künstliche Intelligenz in der Produktion}

% Zugänglichkeit von KI

% zb Matrix über Technik in der KI und Einsatzgebiete in der Produktion -> Wo gibt es Potentiale??


%Methodik



%1. Business Case 2. Daten bereitstellen und aufbereiten  3. Modellentwicklung 4. Modellevaluierung


\section{Energieeffizienz und Nachhaltigkeit}

\section{Kollaborative Roboter}

\section{Beispiel 3}


\chapter{Fazit und Ausblick}

\listoffigures

\clearpage




\clearpage
\addcontentsline{toc}{chapter}{Literaturverzeichnis}

% Nochmal recherchieren wie man IEEE Standard im Literaturverzeichnis umsetzt



\begin{thebibliography}{99}

\bibitem{01}
	%Hier einfügen
	
\bibitem{07}
	Einführung in die Produktion,... 	
 	
 	
\end{thebibliography}

\appendix


\end{document}