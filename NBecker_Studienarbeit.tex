\documentclass[a4paper,12pt, german]{report}
\usepackage[german]{babel}
\usepackage[utf8]{inputenc}
\setcounter{tocdepth}{3}

\usepackage{acro}

\DeclareAcronym{ki}{
  short = KI ,
  long  = künstliche Intelligenz
}



\usepackage{graphicx}
\usepackage{subcaption}
\usepackage{pdflscape}
\usepackage{array}
\usepackage{hhline,float}


\begin{document}
\title{Potentiale beim Einsatz künstlicher Intelligenz in der Produktion}
\author{Nathalie Becker}


\begin{titlepage}
\maketitle
\end{titlepage}


\begin{center}
\textbf{Abstract}
\end{center}
This is the Abstract


\tableofcontents
\printacronyms

\chapter{Einleitung}

Schon seit Jahren sind informatische Errungenschaften und Wirtschaftsbereiche wie die Produktion eng miteinander verknüpft. Längst haben die Unternehmen begriffen, dass der Einsatz von Softwarelösungen einen postiven Effekt auf die Produtivität und die Wirschaftlichkeit haben.
Das Thema der künstlichen Intelligenz ist momentan/schon seit Jahren eines der größten Forschungsfelder in der Informatik. 

RELEVANZ in der Wirtschaft von KI -> auch in der Produktion -> Immer wieder neue Erkenntnisse, viel Forschung -> Potentiale 


\chapter{Stand der Technik}

Das Ziel dieser Arbeit ist es, die Protentiale von künstlicher Intelligenz in der Produktion zu erarbeiten. Zur Erreichung des Ziels wird in diesem Abschnitt die Grundlagen der beiden Themengebiete, künstliche Intelligenz und Produktion eingegangen. Auch der jeweilige aktuelle Stand der Technik wird hier behandelt.

\section{Künstliche Intelligenz}

Kaum eine Technologie bietet Grundlage für mehr Utopien und Dystropien als die künstliche Intelligenz. Filme in denen KI-Roboter die Welt übernehmen, eine Wirtschaft in der durch KI kaum mehr ein Mensch arbeiten muss oder der ..., sind Beispiele dafür. Was KI derzeit wirklich kann und in welchen Gebieten geforscht wird, wird in den kommenden Unterkapiteln aufgeführt.
Künstliche Intelligenz (engl. Artificial Intelligence oder AI) ist ein Forschungzweig der Informatik und beschreibt eher einen Sammelbegriff als konkrete Disziplin oder Technologien. Grob geht es hier um technische Systeme, die selbstständig Lösungen für Probleme finden oder Entscheidungen treffen können. \cite{01} %\cite{10}

Dieses Kapitel der Arbeit klärt den Begriff der künstlichen Intelligenz und beschreibt relevante Ansätze und Technologien aus diesem Forschungsfeld. Anschließend werden aktuelle Anwendungsbereiche behandelt. Mögliche Zukunftsentwicklungen werden unter Zurhilfenahme des Trendsonars des Kompetenzzentrums Öffentliche IT (ÖFIT) aus dem Jahr 2018 zusammengefasst und bewertet. 

\subsection{Grundlagen}

Der Begriff sowie auch die Definition der künstlichen Intelligenz ist in der Literatur uneindeutig. Schon für die Intelligenz an sich existiert keine allgemeingültige Definition und nur umstrittene Methoden zur Messung dieser. Über die letzten Jahrzehnte hat sich sowohl die Beschreibung des Forschungsgebietes, als auch dessen Zielsetzung immer wieder gewandelt. Ihren Ursprung fand die künstliche Intelligenz im Jahr 1955 mit der ersten Defintion von John McCarthy:
\begin{quote}
  Ziel der KI ist es, Maschinen zu entwickeln, die sich verhalten, als verfügten sie über Intelligenz.
\end{quote}
 Mit der Schwierigkeit Intelligenz zu definieren, verliert diese Definition an Aussagenkraft. Eine weitere Definition von Elaine Rich 1983 beschreibt KI wie folgt: 
 \begin{quote}
  Artificial Intelligence is the study of how to make computers do things at which, at the moment, humans are better.
 \end{quote} 
 Damit schafft Rich eine Beschreibung, die sowohl in der Vergangenheit, als auch in der Zukunft Anwendung findet. Beispielsweise sind Menschen bei Erkennen von Objekten, Menschen oder auch den Maschinen noch weit überlegen. Bild- und Spracherkennung sind wichtige Forschungsbereiche der KI.  Ein zentrales Gebiet der Forschung ist demnach auch das maschinelle Lernen, denn vorallem die Lernfähigkeit des Mensch durch Adaptivität stellt sich für Computer als Herausforderung dar.Dahingegen ist die Entwicklung von Schachcomputern nicht mehr relevant, denn diese sind bereits besser als der Mensch. Dennoch geht es bei KI nicht nur um pragmatische praktische Implementierungen intelligenter Verfahren. Ein Teil des Forschungsgebietes ist es auch ein Verständnis für das menschliche Handeln und Schließen, bespielsweise in der Kognitionswissenschaft.
 %\cite{11}

Der Forschungsgegenstand und die Ziele der KI haben sich seit der ihren Anfängen geändert. Während früher ausschließlich an Univerisäten mit Zielen wie der Entwicklung eines Schachcomputers, der dem Mensch überlegen ist, geforscht wurde, stehen heutzutage kommerzielle Anwendungen im Vordergrund. 
Gerade weil KI mittlerweile eine große Bedeutung in der Wirtschaft hat, wird mehr Geld in die Forschung investiert. So forschen nicht nur Universitäten, sondern auch andere Forschungsinstitute und Unternehmen in vielen Bereichen der KI. Auch die Verfügbarkeit von großen Datenmengen zu Trainingszwecken, die durch Verfahren der Daten-Analyse (Big-Data) für KI genutzt werden können, tragen einen großen Teil zur Erfolgen der jüngeren Vergangenheit und Gegenwart bei. 

Grundsätzlich wird zwischen schwacher und starker KI unterschieden. Schwache KI... 
Im Gegensatz dazu wird bei starker KI eine komplette Nachbildung der menschlichen Intelligenz verstanden. Diese ist nicht nur auf ein Gebiet spezialisiert ist, sondern agiert gebietsübergreifend. Das bedeutet, dass Erkenntnisse aus einem Bereich auf andere Bereiche anwenden werden können. In der aktuellen Forschung ist die Entwicklung ein starken Intelligenz jedoch nicht nicht absehbar. 
Im Kotext dieser Arbeit und auch in weiten Teilen der Forschung und Anwendungen wird von schwacher KI gesprochen.

Die meisten KI-Anwendungen sind also auf eine Aufgabe speziell ausgerichtet. 

\subsection{Verfahren und Ansätze des Froschungsfeldes der KI}

Nicht nur die Definition auch eine Segmentierung, Kategorisierung oder Einteilung der Technologien und Ansätze des umfangreichen und höchstdynamischen Forschungsgebietes der künstlichen Intelligenz ist nicht einfach und wird unterschiedlich vorgenommen:
%---Sieben Segmente künstlicher Intelligenz (Maschineller lernen, Robotik, Kybernetik,Expertensysteme, Verarbeitung natürlicher Sprachen, industrielle Bildverarbeitung, Spracherkennung)??( https://www.arnold-it.com/digitalisierung/kuenstliche-intelligenz-ki/)---
%---Teilgebiete aufgeteilt in Handeln, Wahrnhemen und lernen, je 4 untergebiete (https://www.marketinginstitut.biz/blog/kuenstliche-intelligenz/) ---
Diese Arbeit orientiert sich bei der Kategorisierung des Trendsonar des ÖFIT. Dieses unterteilt 28 Technologien und Ansätze in die folgenden drei Bereiche: Lernmethoden, Technologien und Algorithmen, sowie Systeme und Architekturen. Wobei diese Bereich nicht vollständig voneinander zu trennen sind. Im Folgenden werden die relevantesten Ansätze kurz beschrieben. 

\subsubsection{Lernmethoden}
Ein künstliche Intelligenz ist nicht von Beginn an intelligent. Sie muss lernen. Eines der größten Forschungsgebiete von künstlicher Intelligenz ist daher das maschinelle Lernen. Hier wird das technische System mit Daten trainiert wodurch erst die Fähigkeiten der KI entstehen. Es gibt verschiedene Lernmethoden, die für unterschiedliche Anwendungen geeignet sind und genutzt werden. Dennoch bringen auch unkoventionelle Ansätze Durchbrüche in der Forschung. Folgend werden die relevantesten Methoden genannt und kurz beschrieben.

\paragraph{Deep Learning}
\paragraph{Überwachtes Lernen}
\paragraph{Unüberwachtes Lernen}
\paragraph{Bestärkendes Lernen}
\paragraph{Meta-Lernen}


\subsubsection{Technologien und Algorithmen}
KI-Systeme werden aus Bausteinen, wie Technologien und Algorithmen zusammengesetzt. Es werden beispielsweise mathematische Methoden genutzt, um Daten zu klassifizieren und zu verarbeiten oder um Lösungen für Probleme zu finden und zu optimieren. Ein KI-System ensteht also durch die Kombination verschiedener Technologien und Optimierungsmethoden, wodurch spezielle, definierte Aufgaben gelöst werden können. Dies ermöglicht außerdem innovative Funktionalitäten. Einige dieser Bausteine sind hier aufgelistet.

LONG SHORT-TERM MEMORY
GENERATIVE ADVERSARIAL NETWORKS
SUPPORT VECTOR MACHINES

\subsubsection{Systeme und Architekturen}
Wie im oben beschrieben, entstehen KI-Systeme aus der Kombination von KI-Technologien, wodurch eine übergeordnete Funktion realisiert werden kann. Oft werden auch einzelne Komponenten der Systeme und Architekturen ausgetauscht und diese auf dem aktuellen Stand der Technik zu halten, oder um weitere Funktionalitäten zu ermöglichen. Es folgt eine kurze Beschreibung einiger Systeme und Architekturen.

\subsection{Anwendungsbereiche}

wirtschaftliche relevanz, steigend

\subsubsection{Aktuelle Systeme und Technologien}
- zb SAP

\subsubsection{Trendsonar der OEFIT}

- Einschätzung durch Experten

\section{Produktion}

- Prodktion eine der Grundfunktionen eines Industrieunternhemens, nebenBaschaffung und Absatz; abr auch Funanzierung, Forschung unsd Entwcklung, Personal 
- Unternehmensfuntionen je nach Unternehmen unterschiedlich gegliedert; verschiedene Schnittstellen
- Abgrenzung zu Logistik: keine Prod-Logistik, Koordination des Materialflusses und des Informationsflusses zum Wertschöpfungsprozess vom Zulierferer zum Kunden
\cite{07}

- Güter oder Dienstleistungen, die durch eine Kombination von Produktionsfaktoren in einem Transformationsprozess hergestellt werden. Gegensatz zu Verbrauch ( Horst J. Wildermann/Klaus J. Schmidt, Produktion, in: Wolfgang Lück, Lexikon der Betriebswirtschaft, 1990, S. 911)

- Herstellung vs. Bearbeitung vs. Verarbeitung
- Dienstleistungen werden auch produziert: Produktion und Konsum gleichzeitig mit Konsument as externer Produktionsfaktor. 
- Für zu missverstädnissen ob Dienstleistung mit inbegriffen ist, deshalb Leistungserstellung
HIER: NUR PROD ohne Dienstleistungen???

\subsection{Grundlagen und Ziele der Produktion}
% würde die Überschrift Methoden mehr sinn machen??


Gelenkter Einsatz von Proktionsfaktoren zur Herstellung von Gütern oder Erzeugung von Dienstleistungen. Hauptaspekte für erfolgreiche Prosuktion, dh Erreichung der Produktionsziele: Planung, Organisation, Steuerung und Überwachung 
Abbildung: Produktion als Kombinationsprozess: Input -° Throughput -° Output (Abb 1.2)
Abbildung: Produktion im betrieblichen Umfeld (abb1.3)
Produktionsprogramm: bestimmt in welcher Menge und art Produktionsfaktoren eingesetzt/beschafft werden müssen.
Produktion muss finanziert/Investiert werden: Einzahlung erst nach Auszahlung; Anschaffung von Betriebsmitteln (Grundsctück,Gebäude,...)
\cite{07}

HäuftigZiel der Prudktion: monetäre (Gewinn, Umsatz, Maximierung EK,... Kostenminimierung)und nicht-monetäre Ziele (weitere Unterteilung quantitative: Maximierung Marktanteil, Wachstum, Prouktivität / nicht - quanititative Ziele: Sicherung Arbeitsplätze, Vermeidung von Umweltlasten) 
Ökonomische (Gweinn, Kosten, Wirtschaftlichkeit) und nicht ökonomische Ziele (technisch orientierte Ziele: Menge, Qualität, Termine; human- und soziealorientierte Ziele)
Wirtschaftlichkeit ist eine wichtige ökonomische Zielgröße: Verhältnis Output zu Input
Produktivität Verhältnis Ausbringungsmeneg nud Faktoreinsatz (nur eine Produk- und Faktorart) (tech.Ziel)

"Zusammenfassend können als typische Ziele der Produktionswirtschaft geringe Kosten,
ein hoher Output, eine hohe Produktqualität, eine hohe Termineinhaltung und Auslastung
der Fertigungsbereiche (oder einzelner Maschinen) sowie geringe Durchlaufzeiten
genannt werden." Die Ziele fallen in unterschiedliche Planungsbereiche

Produktionsfaktoren: 
Alles, was zur Erstellung der Sach- und dienstleistungen dient, aber auch zur Aufrechterhaltung und Ausbau der Leistungsbereitschaft
In VWL: strak aggregiert-°Arbeit, Boden und Kapital (oder nur Arbeit und Kapital)
In BWL; Abb 1.4 Produktionsfaktorsystem nach Gutenberg; menschl.Arbeit, Betriebsmittel und Werkstoffe (Seite 6 ist alles genau erklärt)
Tab 2.1 Produktionstheorie

\subsection{Produktionsplanung und -steuerung}


\chapter{Aktuelle Anwendungen von künstlicher Intelligenz in der Produktion}

Nachdem im vorangehenden Kapitel die Grundlagen der künstlichen Intelligenz, wie auch der Produktion geklärt wurden, werden diese beiden Themenfelder in den kommenden zwei Abschnitten zusammengeführt. In diesem Kapitel werden aktuelle Anwedungen der künstlichen Intelligenz in der Produktion beschrieben und bewertet. Die Bewertung findet in drei/vier Kategorien statt: Wirtschaftlichkeit, Risiken und Potentiale.

\paragraph{Wirtschaftlichkeit}:  Zur Betrachtung der Wirtschaftlichkeit werden der Aufwand der Implentierung der KI-Technologie in die Produktion und die laufenden Kosten grob den Ersparnissen und den kausalen Erträgen entgegengestellt. 
\paragraph{Risiken}: Entstehen durch das Einsetzen der KI Risiken oder Nachteile für das Unternehmen, wie z.B. Imageverlust oder Gefahr für die Mitarbeiter, werden diese in der Kategorie Risiken bewertet.
\paragraph{Potentiale}: ??? 

Dieser Abschnitt dient auch dazu gegebenenfalls Protentiale dieser Anwedungenzu erkennen und herauszuarbeiten.

% Hier Beispiele über zb Qualitätssicherung, Maintenace,..

% Menschintegrierend, nutzung oder alleine KI/Roboter

% Skalierbarkeit

% Analystische Anwendung, agierende Anwedungen mit Robotern

%Vorgehen zur Entwicklung von KI

\section{Qualitätssicherung}
% Platinenprüfung

\section{Predictive Maintantance}


\section{Produktionsroboter}
% Erkennt, wie Gegestand liegt und kann es aufnhmen und zum nächsten Punkt richtig einstzen



\chapter{Potentiale für künstliche Intelligenz in der Produktion}

Vor allem bei der künstlichen Intelligenz als wichtiges Forschungsfeld der Informatik findet ein stetiger Verbesserungsprozess der Technologien statt. Hierdurch entstehen neue Potentiale für Unternehmen diese Technologien in ihrer Produktion einzusetzen, um Kosten zu sparen

% Zugänglichkeit von KI

% zb Matrix über Technik in der KI und Einsatzgebiete in der Produktion -> Wo gibt es Potentiale??


%Methodik



%1. Business Case 2. Daten bereitstellen und aufbereiten  3. Modellentwicklung 4. Modellevaluierung


\section{Energieeffizienz und Nachhaltigkeit}

\section{Kollaborative Roboter}

\section{Beispiel 3}


\chapter{Fazit und Ausblick}

\listoffigures

\clearpage




\clearpage
\addcontentsline{toc}{chapter}{Literaturverzeichnis}

% Nochmal recherchieren wie man IEEE Standard im Literaturverzeichnis umsetzt


\begin{thebibliography}{99}

\bibitem{01}
	P. Buxmann, H. Schmidt,
	Künstliche Intelligenz: Mit Algorithmen zum wirtschaftlichen Erfolg,
	Springer Gabler,
	2.Auflage,
	2021.
	
\bibitem{02}
	A. Mockenhaupt,
	Digitalisierung und Künstliche Intelligenz in der Produktion: Grundlagen und Anwendung,
	Springer Vieweg,
	2021.

\bibitem{03}
	T. Kaufmann, H. Servatius,
	Das Internet der Dinge und Künstliche Intelligenz als Game Changer: Wege zu einem Management 4.0 und einer digitalen Architektur,
	Springer Vieweg,
	2020.

\bibitem{04} 	
	ChatGPT,
	OpenAI,
	Aufgerufen am 14.01.2023.

\bibitem{05}
	C. Appugliese, P. Nathan, W. S. Roberts
	Agil AI
	O´Reilly Media, Inc.,
	2020.

\bibitem{06}
	GI. Spindler,
	Produktion in Basiswissen Allgemeine Betriebswirtschaftslehre,
	p. 27-46,
	Springer Gabler,
	2022.

\bibitem{07}
	J. Bloech, R. Bogaschewsky, U. Buscher, A. Daub, U. Götze, F. Roland,
	Einführung in die Produktion,
	Springer Gabler, 
	2014.
	
\bibitem{08}
	I. Knappertsbusch, K. Gondlach,
	Arbeitswelt und KI 2030: Herausforderungen und Strategien für die Arbeit von morgen,
	Springer Gabler,
	2022.

\bibitem{09}
	R. Buchkremer, T. Heupel, O. Koch,
	Künstliche Intelligenz in Wirtschaft \& Gesellschaft: Auswirkungen, Herausforderungen \& Handlungsempfehlungen
	Springer Gabler,
	2020.

\bibitem{10}
	C. Welzel, D. Grosch,
	Das ÖFIT-Trendsonar Künstliche Intelligenz,
	Kompetenzzentrum Öffentliche Informationstechnologie,
	2018.

\bibitem{11}
	W. Ertel,
	Grundkurs Künstliche Intelligenz: Eine praxisorientierte Einführung,
	5. Auflage,
	Springer Vieweg,
	2021.

\bibitem{12}
	H. J. Wildermann, K. J. Schmidt,
	Produktion, 
	in: W. Lück, Lexikon der Betriebswirtschaft,
	Verlag moderne Industrie,
	1990.


\bibitem{15}
	T. Zwingmann,
	AI-Powered Business Intelligence,
	O'Reilly Media, Inc.,
	2022.

\bibitem{16}
	U.Walter,
	Künstliche Intelligenz für Dummies (Vortrag),
	Deutsches Museum München,
	https://www.youtube.com/watch?v=B7vCtHvYMyE
	2020.

\bibitem{17}
	D. Sonnet,
	Neuronale Netze kompakt: Vom Perceptron zum Deep Learning,
	Springer Vieweg,
	2022.

\bibitem{18}
	C. Steger, M. Ulrich, C. Wiedemann,
	Machine Vision Algorithms and Applications,
	Wiley-VCH,
	2018.

\bibitem{19}
	A. Jorzig, F. Sarangi,
	Künstliche Intelligenz und Robotik, 
	in: Digitalisierung im Gesundheitswesen,
	Springer,
	2020.

\bibitem{20}
	E. Gutenberg,
	Grundlagen der Betriebswirtschaftslehre, Band 1: Produktion,
	Springer,
	1951.




\end{thebibliography}

\appendix


\end{document}