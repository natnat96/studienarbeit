\documentclass[a4paper,12pt, german]{report}
\usepackage[german]{babel}
\usepackage[utf8]{inputenc}


\usepackage{acro}

\DeclareAcronym{hct}{
  short = HCT ,
  long  = Hämatokrit
}



\usepackage{graphicx}
\usepackage{subcaption}
\usepackage{pdflscape}
\usepackage{array}
\usepackage{hhline,float}


\begin{document}
\title{Potentiale beim Einsatz künstlicher Intelligenz in der Produktion}
\author{Nathalie Becker}


\begin{titlepage}
\maketitle
\end{titlepage}


\begin{center}
\textbf{Abstract}
\end{center}
This is the Abstract


\tableofcontents
\printacronyms

\chapter{Einleitung}

% Hier Inhalt


\chapter{Stand der Technik}

\section{Künstliche Intelligenz}

\subsection{Gebiet 1}

\subsection{Gebiet 2}

\subsection{Gebiet 3}


\section{Produktion}

\subsection{Gebiet 1}

\subsection{Gebiet 2}

\subsection{Gebiet 3}

% Hier noch ein Methodenkapitel??

\chapter{Aktuelle Anwendungen von künstlicher Intelligenz in der Produktion}

% Hier Beispiele über zb Qualitätssicherung, Maintenace,..

\section{Methodik}

\section{Beispiel 2}

\section{Beispiel 3}

\chapter{Potentiale}

% zb Matrix über Technik in der KI und Einsatzgebiete in der Produktion -> Wo gibt es Potentiale??
\section{Beispiel 1}

\section{Beispiel 2}

\section{Beispiel 3}


\chapter{Fazit und Ausblick}

\listoffigures

\clearpage




\clearpage
\addcontentsline{toc}{chapter}{Literaturverzeichnis}

% Nochmal recherchieren wie man IEEE Standard im Literaturverzeichnis umsetzt



\begin{thebibliography}{99}

\bibitem{01}
	%Hier einfügen
	
\bibitem{07}
	Einführung in die Produktion,... 	
 	
 	
\end{thebibliography}

\appendix


\end{document}