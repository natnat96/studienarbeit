

\clearpage
\addcontentsline{toc}{chapter}{Literaturverzeichnis}

% Nochmal recherchieren wie man IEEE Standard im Literaturverzeichnis umsetzt


\begin{thebibliography}{99}

\bibitem{01}
	P. Buxmann, H. Schmidt,
	Künstliche Intelligenz: Mit Algorithmen zum wirtschaftlichen Erfolg,
	Springer Gabler,
	2.Auflage,
	2021.
	
\bibitem{02}
	A. Mockenhaupt,
	Digitalisierung und Künstliche Intelligenz in der Produktion: Grundlagen und Anwendung,
	Springer Vieweg,
	2021.

\bibitem{03}
	T. Kaufmann, H. Servatius,
	Das Internet der Dinge und Künstliche Intelligenz als Game Changer: Wege zu einem Management 4.0 und einer digitalen Architektur,
	Springer Vieweg,
	2020.

\bibitem{04} 	
	ChatGPT,
	OpenAI,
	Aufgerufen am 14.01.2023.

\bibitem{05}
	C. Appugliese, P. Nathan, W. S. Roberts
	Agil AI
	O´Reilly Media, Inc.,
	2020.

\bibitem{06}
	GI. Spindler,
	Produktion in Basiswissen Allgemeine Betriebswirtschaftslehre,
	p. 27-46,
	Springer Gabler,
	2022.

\bibitem{07}
	J. Bloech, R. Bogaschewsky, U. Buscher, A. Daub, U. Götze, F. Roland,
	Einführung in die Produktion,
	Springer Gabler, 
	2014.
	
\bibitem{08}
	I. Knappertsbusch, K. Gondlach,
	Arbeitswelt und KI 2030: Herausforderungen und Strategien für die Arbeit von morgen,
	Springer Gabler,
	2022.

\bibitem{09}
	R. Buchkremer, T. Heupel, O. Koch,
	Künstliche Intelligenz in Wirtschaft \& Gesellschaft: Auswirkungen, Herausforderungen \& Handlungsempfehlungen
	Springer Gabler,
	2020.

\bibitem{10}
	C. Welzel, D. Grosch,
	Das ÖFIT-Trendsonar Künstliche Intelligenz,
	Kompetenzzentrum Öffentliche Informationstechnologie,
	2018.

\bibitem{11}
	W. Ertel,
	Grundkurs Künstliche Intelligenz: Eine praxisorientierte Einführung,
	5. Auflage,
	Springer Vieweg,
	2021.

\bibitem{12}
	H. J. Wildermann, K. J. Schmidt,
	Produktion, 
	in: W. Lück, Lexikon der Betriebswirtschaft,
	Verlag moderne Industrie,
	1990.


\bibitem{15}
	T. Zwingmann,
	AI-Powered Business Intelligence,
	O'Reilly Media, Inc.,
	2022.

\bibitem{16}
	U.Walter,
	Künstliche Intelligenz für Dummies (Vortrag),
	Deutsches Museum München,
	https://www.youtube.com/watch?v=B7vCtHvYMyE
	2020.

\bibitem{17}
	D. Sonnet,
	Neuronale Netze kompakt: Vom Perceptron zum Deep Learning,
	Springer Vieweg,
	2022.

\bibitem{18}
	C. Steger, M. Ulrich, C. Wiedemann,
	Machine Vision Algorithms and Applications,
	Wiley-VCH,
	2018.

\bibitem{19}
	A. Jorzig, F. Sarangi,
	Künstliche Intelligenz und Robotik, 
	in: Digitalisierung im Gesundheitswesen,
	Springer,
	2020.

\bibitem{20}
	E. Gutenberg,
	Grundlagen der Betriebswirtschaftslehre, Band 1: Die Produktion,
	Springer,
	1951.

\bibitem{21}
	S. Kummer, O. Grün, W. Jammernegg,
	Grundzüge der Beschaffung, Produktion und Logistik,
	Pearson,
	2013.

\bibitem{22}
	Bundesministerium für Bildung und Forschung,
	Industrie 4.0,
	https://www.bmbf.de/bmbf/de/forschung/digitale-wirtschaft-und-gesellschaft/industrie-4-0/industrie-4-0,
	aufgerufen am 01.02.2023,
	2016.

\bibitem{23}
	Deloitte Touche Tohmatsu Limited,
	Manufacturing 4.0: Meilenstein, Must-Have oder Millionengrab,
	Deloitte,
	2016.

\bibitem{24}
	B. Hatiboglu, S. Schuler, A. Bildstein, M. Hämmerle, 
	Einsatzfelder von künstlicher Intelligenz im Produktionsumfeld,
	Allianz Industrie 4.0, 
	2019.

\bibitem{25}
	IFR International Federation of Robotics,
	Geschätzter Bestand von Industrierobotern weltweit in den Jahren 2011 bis 2021 (in 1.000 Stück) [Graph],
	In Statista,
	https://de.statista.com/statistik/daten/studie/250212/umfrage/geschaetzter-bestand-von-industrierobotern-weltweit/,
	aufgerufen am 03.02.2023,
	2022.

\bibitem{26}
	IFR International Federation of Robotics,
	Bestand von Industrierobotern weltweit nach Branchen in den Jahren 2018 und 2019 [Graph],
	In Statista,
	https://de.statista.com/statistik/daten/studie/1180158/umfrage/industrieroboter-bestand-weltweit-nach-branchen/,
	aufgerufen am 03.02.2023,
	2020.

\bibitem{27}
	Bitkom Bundesverband Informationswirtschaft, Telekommunikation und neue Medien,
	Nutzen Sie in Ihrem Unternehmen Künstliche Intelligenz im Kontext von Industrie 4.0? (Anteil der ja-Stimmen) [Graph],
	In Statista,
	https://de.statista.com/statistik/daten/studie/990493/umfrage/umfrage-zur-nutzung-kuenstlicher-intelligenz-in-deutschen-industrieunternehmen/,
	aufgerufen am 03.02.2023,
	2020.

\bibitem{28}
	Bitkom Bundesverband Informationswirtschaft, Telekommunikation und neue Medien,
	Was sind aus Sicht Ihres Unternehmens die wichtigsten Vorteile von Künstlicher Intelligenz im Kontext von Industrie 4.0? [Graph],
	In Statista,
	https://de.statista.com/statistik/daten/studie/990505/umfrage/umfrage-zu-vorteilen-kuenstlicher-intelligenz-in-deutschen-industrieunternehmen/,
	aufgerufen am 03.02.2023,
	2020.

\bibitem{29}
	Staufen,
	Welche der folgenden Predictive-Maintenance-Anwendungen nutzen Sie bereits? [Graph],
	In Statista,
	https://de.statista.com/statistik/daten/studie/1078451/umfrage/nutzung-von-predictive-maintenance-anwendungen-in-deutschland/,
	aufgerufen am 03.02.2023,
	2019.

\bibitem{30}
	Bundesministerium für Wirtschaft und Energie: Platform Industrie 4.0,
	Autonome Instandhaltung Industrie 4.0: Selbststeuernde Maschinenüberwachung,
	https://www.plattform-i40.de/IP/Redaktion/DE/Anwendungsbeispiele/141-autonome-instandhaltung-industrie-4-0-selbststeuernde-maschinenueberwachung/beitrag-autonome-instandhaltung-industrie-4-0-selbststeuernde-maschinenueberwachung.html,
	aufgerufen am 03.02.2023,
	2019.

\bibitem{31}
	GESTALT Robotics GmbH,
	KI-gestützte Bildverarbeitung: Qualitätsprüfung, 
	https://www.gestalt-robotics.com/ki-bildverarbeitung, 
	aufgerufen am 03.02.2023,
	ohne Datum.

\end{thebibliography}